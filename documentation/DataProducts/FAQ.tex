% !TEX root = UnofficialGuideToNWTC135mData.tex
\chapter{Frequently Asked Questions}
The following list of common questions and answers will grow as more experience is gained with the towers and the data products.

\section{Data products}
\FAQ[What's the timestamp?]{What does the timestamp represent?}
{The timestamp of the 10-minute data is the start of the 10-minute period. In the high-frequency data, it's the time that measurement was made.}

\FAQ[How do you calculate $x$?]{Can you tell me more about how you calculate $x$?}
{Well, try this:
\begin{enumerate}
\item Start by looking in this document. You can search the PDF for the variable name (without the heights, maybe). 
\item Try looking in the official NREL Technical Report that describes the towers \citep{Clifton_55915}. 
\item Try some of the standard text books such as \cite{Stull_1988_a} or \cite{Garratt_1994_a}. If they have been used as sources for a calculation, they will normally have been referenced.
\item Try searching through the NWTC forums, at \url{https://wind.nrel.gov/forum/wind/}.
\item Still stuck? Please post a question on the forums. 
\end{enumerate}}

\FAQ{Can you add $x$ to your calculations?}
{Create an issue in Github describing what you would like to add and how it would be calculated. It might be possible for us to add the calculation if it doesn't slow down processing and other people can use the result. If the value is a bit more esoteric and can be derived from the existing data, we might suggest you calculate it yourself, rather than introducing a new variable.}

\FAQ[How do I report odd data?]{Data from instrument $x$ or variable $y$ look a little weird. What should I do?}
{If you see odd things in the data, you should check a few things and then report it:
\begin{itemize}
\item Check to see if the automated quality control has picked up that problem. Is a quality flag set for that data?
\item Check to see that the winds are from the west (275\degree $\pm$45\degree). Turbulence quantities or data from the sonic anemometers in particular are affected when the flow is from the east. Also, flow the south may have passed through a turbine. Use the map in \cite{Clifton_55915} for reference.
\item If you are using 10-minute data, only use 10-minute intervals where wind speeds are greater than 3 \ms. Some systems have problems with low wind speeds. Temperature measurements should be OK because the sensors are aspirated, but wind vanes and cups are not always reliable at low wind speeds.
\end{itemize}
If the data haven't been flagged, please send us a note describing which variable you are seeing a problem with and the dates.}

\FAQ[Can I get data in another format?]{Can I get your data in \textit{xx} format?}{We use MATLAB .mat for the raw data and 10-minute data, and comma-delimited ASCII .txt for the 10-minute data currently. We don't plan to support other formats any time soon as we've found that this meets most peoples' needs.}

\section{Comparisons with other data sources}
\FAQ[Can I use the towers to validate my instrument?]{I have a new device that can measure $x$. I'd like to try it up against the met towers. Can I do that?}
{This is something that we can discuss. There are different ways of doing this. A good start is to contact the person who's details are given at \url{http://www.nrel.gov/technologytransfer/tech_partnership_agreements.html} for `wind'.}

\FAQ[Can I use the towers to validate modeling?]{I'd like to try my mesoscale model against your towers. Can I?}
{Please see the question about \textbf{Can I use this data \ldots?} (the short answer is 'yes, if you tell us what you are going to do'). Then, note that the NWTC is nestled under the eastern edge of the Rocky Mountains and Front Range. That said, you are free to use the data for comparison to your mesoscale model, but we warn you that it might shake your confidence in mesoscale modeling...}

\section{Contacts}
\FAQ[What are the `forums'?]{What are the forums you were talking about?}{The forums are message boards hosted at \url{https://wind.nrel.gov/forum/wind/}. Look for the one called `Wind Data'. There might be some answers to your questions there as well.}

\FAQ[Who ya gonna call?]{Who is the point of contact for the data and towers?}{Please direct all queries to the Wind Data forum at \url{https://wind.nrel.gov/forum/wind/}.}
