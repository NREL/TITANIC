% !TEX root = UnofficialGuideToNWTC135mData.tex
\chapter{10-minute summary MATLAB data files \label{s:TenMinuteMatFiles}}
Summary MATLAB data files are generated automatically for every 10-minute period. These file are stored separately from the raw data. The files also have the naming convention \mfile{mmddy\_HH\_MM\_SS\_FFF.mat}. The files contain multiple structures for each of the channels, the sonic anemometer data, and the derived data. 

\section{Variable Structure}
The structure of \mvar{<variable name>} is as follows:

\begin{itemize}
\item \mvar{.val} Mean value during the 10-minute interval, after removing data outside of manufacturer's limits.
\item \mvar{.date} Serial date number of the start of the 10-minute interval.
\item \mvar{.label} Text string to use as a label for charts, etc.
\item \mvar{.units} Text string containing a \LaTeX-formatted string describing the units, e.g. \verb+'m s^{-1}'+.
\item \mvar{.height} The height $z$ above ground [m].
\item \mvar{.npoints} The number of points in the 10-minute interval.
\item \mvar{.flags} Quality-control (QC) codes:
\begin{itemize}
\item QC codes indicating that data are `flagged' (possibly bad) are in the range 1000 to 4999. Reasons for flagging channels include:
\begin{itemize}
\litem{1001} irregular timing. The period between measurements should be 0.05 seconds at a data acquisition rate of 20 Hz. If more than 1\% of data are more than 5\% from the ideal period, this QC code is set.
\litem{1002, 1003} If the number of points within the manufacturer's limits or users' limits is below a threshold set in the configuration file. These threshold values are the \mvar{range} rate (QC code 1002) and the \mvar{accept} rate (QC code 1003).
\item {1004} indicates that the sonic anemometer time series contain less than 95\% of the expected number of records, and thus the data have not been rotated into the mean flow for that interval.
\litem {1006} if the standard deviation drops below 0.01\% of the mean and so a channel is assumed to have a constant value during the measurement interval.
\litem {20$nn$} if a channel is flagged because it is linked with another channel that has been flagged, where $nn$ is the number of the channel that was flagged.

\end{itemize}
\item QC codes indicating that channels or data have failed are greater than 5000. Reasons for marking channels as failed include:
\begin{itemize}
\litem {5001} if a channel is empty.
\litem {5002} if all data in a channel have known `bad' values, e.g. -999.
\litem {5003} if all data in a channel are not-a-number (NaN).
\litem {5004} if the boom speed exceeds 0.1 m/s at any time during the 10 minute interval.
\litem {5005} if the channel is affected by a known outage.
\litem{60$nn$} if a channel fails because it is linked with another channel that has failed, where $nn$ is the number of the channel that failed.
\end{itemize}
\end{itemize}
\end{itemize}

For brevity, usually only the output value in \mvar{<variable\_name>.val} and the quality control output to \mvar{<variable\_name>.flags} will be described in the following sections.

\section{Metadata}
%% DATA FILES
\subsection{Data file records}
\begin{itemize}
\item \mvar{Data\_File\_Records} The number of records in the data file (rows). 
\item \mvar{Data\_File\_Valid\_Records} The number of the first record at which data import fails.
\end{itemize}

\section{Instrument data}
%% CHANNEL MEANS
\subsection{Channel means and standard deviations}
\begin{itemize}
\item \mvar{Raw\_<Channel\_name>\_<z>m\_mean} Mean value of all channel data within the manufacturer's limits from the 10-minute interval. 
\item \mvar{Raw\_<Channel\_name>\_<z>m\_sdev} Standard deviation of all channel data within the manufacturer's limits from the 10-minute interval.
\end{itemize}

%%%%%%%%
%% CUPS %%
%%%%%%%%
\section{Cup wind speed and turbulence intensity}
\begin{itemize}
\item \mvar{Wind\_Speed\_Cup\_<z>m.val} Mean value of all cup wind speeds within the manufacturer's limits from the 10-minute interval. 
\item \mvar{Wind\_Speed\_Cup\_<z>m.flags} Quality flags for the wind speed are inherited from the channel data.
\item \mvar{Ti\_Cup\_<z>m.val} Ratio of standard deviation to mean (percentage) of all cup wind speeds within the manufacturer's limits from the 10-minute interval.
\item \mvar{Ti\_Cup\_<z>m.flags} Quality flags for the turbulence intensity are inherited from the mean wind speed at the heights at which the turbulence intensity is calculated.
\end{itemize}

%%%%%%%%%
%% SONICS %%
%%%%%%%%%
\section{Sonic Anemometer data}
If any of the individual sonic velocity component channels or temperature channel are coded with the codes \textbf{1006} (low standard deviation), \textbf{5001} (empty data channel), \textbf{5002} (all bad values) or \textbf{5003} (all NaNs), these quality codes are inherited. Other quality codes are ignored.

%% SONIC HORIZONTAL WIND SPEED %%
\subsection{Horizontal wind speed (sonic anemometer)\label{s:MDP_S_WS}}\index{output data!wind speed!sonic anemometer} \index{output structure!Wind\_Speed\_Horizontal\_Sonic\_<z>m@\textsl{Wind\_Speed\_Horizontal\_Sonic\_<z>m}}
\begin{itemize}
\item \mvar{Wind\_Speed\_Horizontal\_Sonic\_<z>m.val} Horizontal wind speed at height $z$ [m s\textsuperscript{-1}].
\item \mvar{Wind\_Speed\_Horizontal\_Sonic\_<z>m.flags} QC flags are defined depending on the amount of data, the boom motion and individual channel status.
\end{itemize}

%% SONIC CUP-EQUIVALENT WIND SPEED
\subsection{Cup-equivalent wind speed (sonic anemometer)}\index{output data!wind speed!sonic anemometer}\index{output structure!Wind\_Speed\_CupEq\_Sonic\_<z>m@\textsl{Wind\_Speed\_Advection\_Sonic\_<z>m}}
\begin{itemize}
\item \mvar{Wind\_Speed\_CupEq\_Sonic\_<z>m.val} Cup-equivalent wind speed at height $z$ [m s\textsuperscript{-1}].
\item \mvar{Wind\_Speed\_CupEq\_Sonic\_<z>m.flags} Quality control codes used for the cup-equivalent wind speed are inherited from the horizontal wind speed (Section \ref{s:MDP_S_WS}).
\end{itemize}

%% SONIC CUP-EQUIVALENT Ti
\subsection{Cup-equivalent turbulence intensity (sonic anemometer)}
\begin{itemize}
\item \mvar{Ti\_CupEq\_Sonic\_<z>m.val} The cup-equivalent turbulence intensity of the streamwise velocity component at height $z$ [\%]. \index{output data!velocity statistics!sonic anemometer} \index{output structure!Ti\_Sonic\_<z>m@\textsl{Ti\_Sonic\_<z>m}}
\item \mvar{Ti\_CupEq\_Sonic\_<z>m.flags} Turbulence statistics are only calculated if the number of data points exceeds the target percentage given in the configuration file by the parameter \mvar{tower.sonicrotate}. If there is not enough data, the quality control code \textbf{1004} is recorded. Other quality control codes are inherited from the horizontal wind speed (Section \ref{s:MDP_S_WS}).
\end{itemize}

%% SONIC TOTAL WIND SPEED
\subsection{Total wind speed (sonic anemometer)}\index{output data!wind speed!sonic anemometer}\index{output structure!Wind\_Speed\_Total\_Sonic\_<z>m@\textsl{Wind\_Speed\_Total\_Sonic\_<z>m}}
\begin{itemize}
\item \mvar{Wind\_Speed\_Total\_Sonic\_<z>m.val} Total wind speed at height $z$ [m s\textsuperscript{-1}].
\item \mvar{Wind\_Speed\_Total\_Sonic\_<z>m.flags} Quality control codes used for the total wind speed are inherited from the horizontal wind speed (Section \ref{s:MDP_S_WS}).
\end{itemize}

%% FLOW ANGLE
\subsection{Flow angle (sonic anemometer)}\index{output data!inflow angle!sonic anemometer}\index{output structure!Wind\_Inflow\_Angle\_Sonic\_<z>m@\textsl{Wind\_Inflow\_Angle\_Sonic\_<z>m}}
\begin{itemize}
\item \mvar{Wind\_Inflow\_Angle\_Sonic\_<z>m.val} Inflow angle at height $z$. A positive angle implies a mean vertical velocity above zero (rising flow).
\item \mvar{Wind\_Inflow\_Angle\_Sonic\_<z>m.flags} Quality control codes used for the inflow angle are inherited from the horizontal wind speed (Section \ref{s:MDP_S_WS}).
\end{itemize}

%% SONIC ADVECTION WIND SPEED
\subsection{Advection wind speed (sonic anemometer)}\index{output data!wind speed!sonic anemometer}\index{output structure!Wind\_Speed\_Advection\_Sonic\_<z>m@\textsl{Wind\_Speed\_Advection\_Sonic\_<z>m}}
\begin{itemize}
\item \mvar{Wind\_Speed\_Advection\_Sonic\_<z>m.val} Advection wind speed at height $z$ [m s\textsuperscript{-1}].
\item \mvar{Wind\_Speed\_Advection\_Sonic\_<z>m.flags} Quality control codes used for the advection wind speed are inherited from the horizontal wind speed (Section \ref{s:MDP_S_WS}).
\end{itemize}

%% SONIC STANDARD DEVIATIONS
\subsection{Standard deviations of velocities and temperatures (sonic anemometer)}
\begin{itemize}
\item \mvar{Sigma\_u\_Sonic\_<z>m.val} The standard deviation of the streamwise velocity component at height $z$ [m s\textsuperscript{-1}].
\index{output data!velocity statistics!sonic anemometer} \index{output structure!Sigma\_u\_Sonic\_<z>m@\textsl{Sigma\_u\_Sonic\_<z>m}}
\item \mvar{Sigma\_v\_Sonic\_<z>m.val} The standard deviation of the lateral velocity component at height $z$ [m s\textsuperscript{-1}].
\index{output data!velocity statistics!sonic anemometer} \index{output structure!Sigma\_v\_Sonic\_<z>m@\textsl{Sigma\_v\_Sonic\_<z>m}}
\item \mvar{Sigma\_w\_Sonic\_<z>m.val} The standard deviation of the vertical velocity component at height $z$ [m s\textsuperscript{-1}].
\index{output data!velocity statistics!sonic anemometer} \index{output structure!Sigma\_w\_Sonic\_<z>m@\textsl{Sigma\_w\_Sonic\_<z>m}}
\item \mvar{Sigma\_T\_Sonic\_<z>m.val} The standard deviation of the temperature at height $z$ [K].
\index{output data!temperature statistics!sonic anemometer} \index{output structure!Sigma\_T\_Sonic\_<z>m@\textsl{Sigma\_T\_Sonic\_<z>m}}
\item \mvar{Sigma\_<u,v,w,T>\_Sonic\_<z>m.flags} Turbulence statistics are only calculated if the number of data points exceeds the target percentage given in the configuration file by the parameter \mvar{tower.sonicrotate}. If there is not enough data, the quality control code \textbf{1004} is recorded. Other quality control codes used are inherited from the horizontal wind speed (Section \ref{s:MDP_S_WS}).
\end{itemize}

%% SONIC FRICTION VELOCITY
\subsection{Friction velocity (sonic anemometer)}\index{output data!friction velocity!sonic anemometer} \index{output structure!ustar\_Sonic\_<z>m@\textsl{ustar\_Sonic\_<z>m}}
\begin{itemize}
\item \mvar{ustar\_Sonic\_<z>m.val} The friction velocity at height $z$ m [m s\textsuperscript{-1}].
\item \mvar{ustar\_Sonic\_<z>m.flags} Friction velocity is only calculated if the number of data points exceeds the target percentage given in the configuration file by the parameter \mvar{tower.sonicrotate}. If there is not enough data, the quality control code \textbf{1004} is recorded. Other quality control codes are inherited from the horizontal wind speed (Section \ref{s:MDP_S_WS}).
\end{itemize}

%% CONVECTIVE TEMPERATURE SCALE
\subsection{Convective temperature scale (sonic anemometer)} \index{output data!convective temperature scale} \index{output structure!Tstar\_Sonic\_<z>m\_mean@\textsl{Tstar\_Sonic\_<z>m\_mean}}
\begin{itemize}
\item \mvar{Tstar\_Sonic\_<z>m\_mean.val} The convective temperature scale at height $z$ [k].
\end{itemize}

%% TKE
\subsection{Turbulent kinetic energy (sonic anemometer)}\index{output data!turbulent kinetic energy} \index{output structure!TKE\_Sonic\_<z>m\_mean@\textsl{TKE\_Sonic\_<z>m\_mean}}\index{output structure!TKE\_Sonic\_<z>m\_peak@\textsl{TKE\_Sonic\_<z>m\_peak}}
\begin{itemize}
\item \mvar{TKE\_Sonic\_<z>m\_mean.val} The mean turbulent kinetic energy ($\overline{\textrm{TKE}}$) at height $z$ [m\textsuperscript{2} s\textsuperscript{-2}]. 
\item \mvar{TKE\_Sonic\_<z>m\_peak.val} The maximum turbulent kinetic energy ($\textrm{TKE}(t)$) at height $z$ during the 10-minute interval [m\textsuperscript{2} s\textsuperscript{-2}].
\item \mvar{TKE\_Sonic\_<z>m\_<mean,peak>.flags} Turbulence statistics are only calculated if the number of data points exceeds the target percentage given by the parameter \mvar{tower.sonicrotate} (Section \ref{s:20HzMetadata}). If there is not enough data, the quality control code \textbf{1004} is recorded. Other quality control codes used are inherited from the horizontal wind speed (Section \ref{s:MDP_S_WS}).
\end{itemize}

%% CTKE
\subsection{Coherent TKE (sonic anemometers)}\index{output data!coherent turbulent kinetic energy} \index{output structure!CTKE\_Sonic\_<z>m\_mean@\textsl{CTKE\_Sonic\_<z>m\_mean}}\index{output structure!CTKE\_Sonic\_<z>m\_peak@\textsl{CTKE\_Sonic\_<z>m\_peak}}
\begin{itemize}
\item \mvar{CTKE\_Sonic\_<z>m\_mean.val} The RMS coherent turbulent kinetic energy at height $z$ [m\textsuperscript{2} s\textsuperscript{-2}]. 
\item \mvar{CTKE\_Sonic\_<z>m\_peak.val} The maximum coherent turbulent kinetic energy at height $z$ during the 10-minute interval [m\textsuperscript{2} s\textsuperscript{-2}].
\item \mvar{CTKE\_Sonic\_<z>m\_<mean,peak>.flags} CTKE is only calculated if the number of data points exceeds the target percentage given by the parameter \mvar{tower.sonicrotate}. If there is not enough data, the quality control code \textbf{1004} is recorded. Other quality control codes are inherited from the horizontal wind speed (Section \ref{s:MDP_S_WS}).
\end{itemize}

%% INTEGRAL LENGTH SCALES
\subsection{integral length scales (sonic anemometers)}\index{output data!turbulence length scale} \index{output structure!L\_u\_Sonic\_<z>m@\textsl{L\_u\_Sonic\_<z>m}} \index{output structure!L\_v\_Sonic\_<z>m@\textsl{L\_v\_Sonic\_<z>m}}\index{output structure!L\_w\_Sonic\_<z>m@\textsl{L\_w\_Sonic\_<z>m}}
Several different integral length scales are defined, after \cite{Pichugina_2008_a}.
\begin{itemize}
\item \mvar{L\_integral\_<u,v,w>\_Sonic\_<z>m.val} The integral of the autocorrelation function from 0 to infiinty at height $z$ [m].
\item \mvar{L\_zc\_<u,v,w>\_Sonic\_<z>m.val} The integral of the autocorrelation function from 0 to the zero crossing at height $z$ [m].
\item \mvar{L\_sz\_<u,v,w>\_Sonic\_<z>m.val} The integral length scale from $t= 0$ to the zero-crossing time at height $z$ [m]. % UNSATISFACTORY!
\item \mvar{L\_Peak\_<u,v,w>\_Sonic\_<z>m.val} The length scale associated with the peak of the spectrum $fS(f)$ at height $z$ [m].
\item \mvar{L\_Kaim\_<u,v,w>\_Sonic\_<z>m.val} The length scale associated with a fit to the spectrum assuming the spectrum has a Kaimal spectral, at height $z$ [m].
\item \mvar{L\_<integral, zc, sz, Peak, Kaim>\_<u,v,w>Sonic\_<z>m.flags} Turbulence time and length scales are only calculated if the number of data points exceeds the target percentage given in the configuration file by the parameter \mvar{tower.sonicrotate}. If there is not enough data, the quality control code \textbf{1004} is recorded. Other quality control codes used are inherited from the horizontal wind speed (Section \ref{s:MDP_S_WS}).
\end{itemize}

%% STRUCTURE FUNCTIONS
\subsection{Structure functions of velocity and temperature (sonic anemometers)}\index{output data!structure function parameter}\index{output data!Cv2} \index{output structure!CV2m\_Sonic\_<z>m@\textsl{CV2m\_Sonic\_<z>m}}\index{output data!structure function parameter}\index{output data!CT2} \index{output structure!CT2m\_Sonic\_<z>m@\textsl{CT2m\_Sonic\_<z>m}}
\begin{itemize}
\item \mvar{CV2m\_Sonic\_<z>m.val} The median value of $C_{V^2}$ at height $z$ [m\textsuperscript{2} s\textsuperscript{-3}]. 
\item \mvar{CT2m\_Sonic\_<z>m.val} The median value of $C_{T^2}$ at height $z$ [k\textsuperscript{2} m\textsuperscript{2/3}]. % CHECK UNITS 
\end{itemize}

The structure function parameters $C_{V^2}$ and $C_{T^2}$ are not quality controlled.

%% DISSIPATION RATE
\subsection{Dissipation rate (sonic anemometer)}\index{output data!dissipation rate}\index{output structure!Drate\_SF\_Sonic\_<z>m@\textsl{Drate\_SF\_Sonic\_<z>m}}
\begin{itemize}
\item \mvar{Drate\_SF\_Sonic\_<z>m.val} The dissipation rate $\epsilon$ at height $z$ .
\item \mvar{Drate\_SF\_Sonic\_<z>m.flags} The dissipation rate is only calculated if the number of data points exceeds the target percentage given in the configuration file by the parameter \mvar{tower.sonicrotate}. If there is not enough data, the quality control code \textbf{1004} is recorded. Other quality control codes used are inherited from the horizontal wind speed (Section \ref{s:MDP_S_WS}).
\end{itemize}

%%%%%%%%%%%%%
%% DERIVED DATA %%
%%%%%%%%%%%%%
\section{Derived Data}
%% VANE WIND DIRECTION
\subsection{Wind direction (cups and vanes)}\index{output data!wind speed!cup anemometer}\index{output structure!Wind\_Speed\_Cup\_<z>m\_mean@\textsl{Wind\_Speed\_Cup\_<z>m\_mean}}\index{output data!wind direction!mean}\index{output structure!Wind\_Direction\_Vane\_<z>m@\textsl{Wind\_Direction\_Vane\_<z>m}}\index{output structure!Wind\_Direction\_Vane\_<z>m\_sdev@\textsl{Wind\_Direction\_Vane\_<z>m\_sdev}}\index{wind direction!standard deviation!cup}\index{output data!wind direction!standard deviation}
\begin{itemize}
\item \mvar{Wind\_Direction\_Vane\_<z>m.val} Mean wind direction measured by the vane during the 10 minute interval [\degree].
\item \mvar{Wind\_Direction\_Vane\_<z>m\_sdev.val} Standard deviation of the wind direction measured by the vane during the 10 minute interval [\degree].
\item \mvar{....flags} Quality flags for the wind speed and wind direction are inherited from the time-resolved wind speed and direction measurement channels. For example, if there are insufficient data in the wind speed time series, the mean wind speed will also be flagged as having too few data points with the QC code \textbf{1002} or \textbf{1003}.
\end{itemize}

%% CUP SHEAR
\subsection{Power law velocity profile exponent (cups)}\index{output data!wind shear}\index{output structure!Wind\_Shear\_z2\_z2m@\textsl{Wind\_Shear\_z2\_z2m}}
\begin{itemize}
\item \mvar{Wind\_Shear\_z2\_z2m.val} Power law exponent calculated from the mean velocities at heights $z1$ and $z2$ [-]. 
\item \mvar{Wind\_Shear\_z2\_z2m.flags} Quality flags for the wind shear are inherited from the mean wind speed at the heights over which the shear is calculated. Data area also flagged if the best fit does not converge.
\end{itemize}

%% CUP USTAR AND Z0
\subsection{Log law friction velocity and roughness length} \index{output data!friction velocity} \index{output structure!Friction\_velocity\_cup\_<z1>\_<z2>m@\textsl{Friction\_velocity\_cup\_<z1>\_<z2>m}}\index{roughness length} \index{output data!roughness length} \index{output structure!Roughness\_Length\_cup\_<z1>\_<z2>m@\textsl{Roughness\_Length\_cup\_<z1>\_<z2>m}}
\begin{itemize}
\item \mvar{Friction\_velocity\_cup\_<z1>\_<z2>m.val} Friction velocity [m s\textsuperscript{-1}]. 
\item \mvar{Friction\_velocity\_cup\_<z1>\_<z2>m.flags} Quality codes for the friction velocity are inherited from codes for the wind speed and wind direction at the heights over which the friction velocity is calculated.
\item \mvar{Roughness\_Length\_cup\_<z1>\_<z2>m.val} Roughness length [m].
\item \mvar{Roughness\_Length\_cup\_<z1>\_<z2>m.flags} Quality codes for the roughness length are inherited from codes for the wind speed and wind direction at the heights over which the friction velocity is calculated.
\end{itemize}

%% WIND VEER
\subsection{Wind veer (cups)}\index{output data!wind veer} \index{output structure!Wind\_Veer\_<z1>\_<z2>m@\textsl{Wind\_Veer\_<z1>\_<z2>m}}
\begin{itemize}
\item \mvar{Wind\_Veer\_<z1>\_<z2>m.val} Wind veer between heights $z1$ and $z2$. If multiple heights are defined, for example, across the rotor disk of the turbine, the largest difference is used [\degree]. 
\item \mvar{Wind\_Veer\_<z1>\_<z2>m.flags} Quality codes for the wind veer are inherited from the mean wind direction at the heights over which the veer is calculated.
\end{itemize}

%% RAIN
\subsection{Rain}\index{output data!precipitation} \index{output structure!Raw\_PRECIP\_INTEN\_mean.val@\textsl{Raw\_PRECIP\_INTEN\_mean.val}}
The channel number of the rain sensor is given in the configuration file (\mvar{tower.precipsensor = c}, see Section \ref{s:20HzMetadata}).
\begin{itemize}
\item \mvar{Raw\_PRECIP\_INTEN\_mean.val} Mean during the measurement interval.
\end{itemize}

%% AIR TEMPERATURE
\subsection{Air temperature}\index{output data!air temperature} \index{output structure!Air\_Temperature\_<z>m@\textsl{Air\_Temperature\_<z>m}}
\begin{itemize}
\item \mvar{Air\_Temperature\_<z>m.val} Air temperature at height $z$ [\degree C]. 
\item \mvar{Air\_Temperature\_<z>m.flags} The air temperature inherits the quality codes of the tower-base temperature measurement and the differential temperature measurements.
\end{itemize}

%% RELATIVE HUMIDITY
\subsection{Relative humidity}\index{output data!relative humidity} \index{output structure!Relative\_Humidity\_<z>m@\textsl{Relative\_Humidity\_<z>m}}
\begin{itemize}
\item \mvar{Relative\_Humidity\_<z>m.val} The relative humidity at height $z$ [\%]. 
\item \mvar{Relative\_Humidity\_<z>m.flags} The relative humidity inherits the quality codes of the relative and saturation vapor pressure.
\end{itemize}

%% AIR PRESSURE
\subsection{Air pressure}\index{output data!air pressure} \index{output structure!Air\_Pressure\_<z>m@\textsl{Air\_Pressure\_<z>m}}
\begin{itemize}
\item \mvar{Air\_Pressure\_<z>m.val} The air pressure at height $z$ [mBar]. 
\item \mvar{Air\_Pressure\_<z>m.flags}  The pressure at each height inherits the quality codes of the ground pressure and pressure gradient.
\end{itemize}

%% POTENTIAL TEMPERATURE
\subsection{Potential temperature}\index{output data!potential temperature} \index{output structure!Potential\_Temperature\_<z>m@\textsl{Potential\_Temperature\_<z>m}}
\begin{itemize}
\item \mvar{Potential\_Temperature\_<z>m.val} The potential temperature at height $z$.
\item \mvar{Potential\_Temperature\_<z>m.flags} The potential temperature inherits the quality codes from the pressure profile and calculated air temperature profile.
\end{itemize}

%% VIRTUAL POTENTIAL TEMPERATURE
\subsection{Virtual potential temperature}\index{output data!virtual potential temperature} \index{output structure!Virtual\_Potential\_Temperature\_<z>m@\textsl{Virtual\_Potential\_Temperature\_<z>m}}
\begin{itemize}
\item \mvar{Virtual\_Potential\_Temperature\_<z>m.val} The 10-minute mean virtual potential temperature at height $z$ [\degree]. 
\item \mvar{Virtual\_Potential\_Temperature\_<z>m.flags} The virtual potential temperature inherits the quality codes from the local specific humidity, air temperature and pressure.
\end{itemize}

%% GRADIENT RICHARDSON
\subsection{Gradient Richardson number}\index{output data!Richardson number!gradient} \index{output structure!Ri\_grad\_<z1>\_<z2>m@\textsl{Ri\_grad\_<z1>\_<z2>m}}
\begin{itemize}
\item \mvar{Ri\_grad\_<z1>\_<z2>m.val} Gradient Richardson number between heights $z_1$ and $z_2$ [-].

Where the height interval scans multiple measurement heights, all of those heights will be used in the name, e.g \mvar{Ri\_grad\_<z1>\_<z2>\_<z3>\_<z4>m.val}.
\item \mvar{Ri\_grad\_<z1>\_<z2>m.height} Heights $z_1$ and $z_2$ [m]. 
\item \mvar{Ri\_grad\_<z1>\_<z2>m.flags} The gradient Richardson Number inherits the quality codes from the wind speed and direction, and the virtual potential temperature data that were used.
\end{itemize}

%% SPEED GRADIENT RICHARDSON
\subsection{Speed Gradient Richardson number}\index{output data!Richardson number!speed}\index{output structure!Ri\_WS\_<z1>\_<z2>m@\textsl{Ri\_WS\_<z1>\_<z2>m}}
\begin{itemize}
\item \mvar{Ri\_WS\_<z1>\_<z2>m.val} Speed Richardson Number between heights $z_1$ and $z_2$ [-].
\item \mvar{Ri\_WS\_<z1>\_<z2>m.height} Heights $z_1$ and $z_2$ [m].
\item \mvar{Ri\_WS\_<z1>\_<z2>m.flags} The Richardson number inherits quality control flags from the individual speed and direction sensors that are used in this calculation. 
\end{itemize}

%% BRUNT-VAISALA FREQ
\subsection{Brunt-Vaisala Frequency}\index{output data!Brunt-Vaisala Frequency}\index{output structure!BruntVaisala\_<z1>\_<z2>m@\textsl{BruntVaisala\_<z1>\_<z2>m}}
\begin{itemize}
\item \mvar{BruntVaisala\_<z1>\_<z2>m.val} Brunt-Vaisala Frequency ($N$) between heights $z1$ and $z2$ [s\textsuperscript{-1}]. 
\item \mvar{BruntVaisala\_<z1>\_<z2>height} Heights $z1$ and $z2$ [m].
\item \mvar{BruntVaisala\_<z1>\_<z2>m.flags} If the speed Richardson number $Ri_S$ exceeds $\pm$10, the Brunt-Vaisala frequency is set to NaN and the flag \textbf{1007} is set.
\end{itemize}

%% HEAT FLUX
\subsection{Heat flux (sonic anemometers)} \index{output data!heat flux} \index{output structure!wT\_Sonic\_<z>m\_mean@\textsl{wT\_Sonic\_<z>m\_mean}}\index{output structure!Heat\_flux\_Sonic\_<z>m@\textsl{Heat\_flux\_Sonic\_<z>m}}
\begin{itemize}
\item \mvar{wT\_Sonic\_<z>m\_mean.val} The mean value of $w^\prime T_s^\prime$ at height $z$ [\ms\ K].
\item \mvar{wT\_Sonic\_<z>m\_mean.flags} The mean of $w^\prime T_s^\prime$ is only calculated if the number of data points exceeds the target percentage given in the configuration file by the parameter \mvar{tower.sonicrotate}. If there is not enough data, the quality control code \textbf{1004} is recorded.
\item \mvar{Heat\_flux\_Sonic\_<z>m.val} The heat flux $Q$ at height $z$ [W m\textsuperscript{-2}]. 
\item \mvar{Heat\_Flux\_Sonic\_<z>m.flags} The heat flux is only calculated if the number of data points exceeds the target percentage given in the configuration file by the parameter \mvar{tower.sonicrotate}. If there is not enough data, the quality control code \textbf{1004} is recorded. The heat flux requires the air density $\rho$ and so is only calculated if there is a thermistor temperature profile. Other quality control codes are inherited from the horizontal wind speed (Section \ref{s:MDP_S_WS}).
\end{itemize}

%% MONIN-OBUKHOV LENGTH
\subsection{Monin-Obukhov length}\index{output data!Monin-Obukhov length}\index{output structure!MO\_Length\_Sonic\_<z>m@\textsl{MO\_Length\_Sonic\_<z>m}}\index{output data!Monin-Obukhov length!normalized} \index{output structure!zover\_MO\_Length\_Sonic\_<z>m@\textsl{zover\_MO\_Length\_Sonic\_<z>m}}
\begin{itemize}
\item \mvar{MO\_Length\_Sonic\_<z>m.val} The Monin-Obukhov length $L$ at height $z$ [m].
\item \mvar{zover\_MO\_Length\_Sonic\_<z>m.val} The ratio $\zeta = z / L$, which is the Monin-Obukhov length, normalized by the measurement height $z$ [-].
\item \mvar{MO\_Length\_Sonic\_<z>m.flags} The Monin-Obukhov length $L$ is only calculated if the number of data points exceeds the target percentage given in the configuration file by the parameter \mvar{tower.sonicrotate}. If there is not enough data, the quality control code \textbf{1004} is set. Other quality control codes used are inherited from the horizontal wind speed (Section \ref{s:MDP_S_WS}).
\item \mvar{zover\_MO\_Length\_Sonic\_<z>m.flags} The ratio $z/L$ is only calculated if $L$ is calculated. If there is not enough data to calculate $L$, the quality control code \textbf{1004} is set. Other quality control codes used are inherited from the horizontal wind speed (Section \ref{s:MDP_S_WS}). Quality control codes can also be inherited from the virtual potential temperature.
\end{itemize}

